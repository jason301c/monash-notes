% NOTES TEMPLATE V1.0 ------------------------------------------------------
\documentclass[12pt,a4paper]{article}

% Packages
\usepackage{amsmath, amssymb, amsthm}
\usepackage{graphicx}
\usepackage{algorithm} 
\usepackage{algpseudocode}
\usepackage{tikz}
\usepackage{fancyhdr}
\usepackage{hyperref}
\usepackage{enumitem}
\usepackage[margin=1in]{geometry}

% Theorem environments
\theoremstyle{plain}
\newtheorem{theorem}{Theorem}[section]
\newtheorem{lemma}[theorem]{Lemma}
\newtheorem{proposition}[theorem]{Proposition}
\newtheorem{corollary}[theorem]{Corollary}
\theoremstyle{definition}
\newtheorem{definition}[theorem]{Definition}
\newtheorem{example}[theorem]{Example}
\theoremstyle{remark}
\newtheorem{remark}[theorem]{Remark}

% Important highlight command
\newcommand{\important}[1]{\textcolor{blue}{#1}}
\usepackage{xcolor}

% Header and footer setup
\pagestyle{fancy}
\fancyhf{}
\rhead{FIT2014 - Theory of Computation} % Course name goes here
\lhead{Lecture 02} % Lecture number goes here
\cfoot{\thepage}

% Title information
\title{Lecture 2 - Propositional Logic} % Note title
\author{Jason Alexander} % Your name
\date{\today}

\begin{document}

\maketitle
\tableofcontents
\newpage

% NOTES TEMPLATE ENDS HERE --------------------------------------------------------

\section{Propositional Logic}
\subsection{Propositions}
\begin{definition}
A \textbf{proposition} is a statement that is either true or false, but not both.
\end{definition}

\subsection{Logical Operations}
A nice way to represent operations on propositions is by using truth tables which lists all the possible values of arguments and states, for each, the resulting value.
% Negation
\begin{definition}
The \textbf{negation} of a proposition $p$ is denoted by $\neg p$ and is true if $p$ is false, and false if $p$ is true.
\end{definition}

% Conjunction
\begin{definition}
The \textbf{conjunction} (and) of propositions $p$ and $q$ is denoted by $p \land q$ and is true if both $p$ and $q$ are true, and false otherwise.
This is often called
\end{definition}

% Disjunction
\begin{definition}
The \textbf{disjunction} (inclusive OR) of propositions $p$ and $q$ is denoted by $p \lor q$ and is true if at least one of $p$ and $q$ is true, and false otherwise.
\end{definition}

% De Morgan Laws
\subsection{De Morgan's Laws}
Conjunction and disjunction are related by De Morgan's laws.
\begin{theorem}
\textbf{De Morgan's Laws}:
\begin{align*}
\neg (p \land q) &\equiv \neg p \lor \neg q \\
\neg (p \lor q) &\equiv \neg p \land \neg q
\end{align*}
\end{theorem}

There is also an intuitive way to think about it. A fun example: ``I will not eat my vegetables and I will not eat my fruits'' is equivalent to ``I will not eat my vegetables or fruits''.

You can also prove De Morgan's laws using truth tables.

% Implication
\subsection{Implication}
\begin{definition}
The \textbf{implication} of propositions $p$ and $q$ is denoted by $p \rightarrow q$ and is true if $p$ is false or $q$ is true, and false otherwise.
\end{definition}

An example: ``If it is raining, then I will take an umbrella''. If I do take an umbrella even if it doesn't rain, the statement is still true.

Implication is not neccessarily causation. For example, ``If I am in Istanbul, then I am in Turkey'' is true even if I am not in Istanbul.

Equivalent words: "if...then", "only if", "implies", "is sufficient for", "is necessary for".

% Biconditional
\subsection{Biconditional / bi-implication}
\begin{definition}
The \textbf{biconditional} of propositions $p$ and $q$ is denoted by $p \leftrightarrow q$ and is true if $p$ and $q$ have the same truth value, and false otherwise.
\end{definition}

Equivalent words: "if and only if", "iff", "is equivalent to". An example of this is that a right-angle triangle must satisfy $a^2 + b^2 = c^2$

% Tautologies and logical equivalence
\subsection{Tautologies and Logical Equivalence}
\begin{definition}
A \textbf{tautology} is a proposition that is always true, regardless of the truth values of its variables.
\end{definition}

In other words, the right hand column of the truth table is all true.

\begin{definition}
    Two statements are \textbf{logically equivalent} if their truth tables are identical.
\end{definition}

If $P$ and $Q$ are logically equivalent, $P \leftrightarrow Q$ is a tautology.

% Laws of Boolean Algebra
\subsection{Laws of Boolean Algebra}
A full listing of laws of Boolean algebra:

\vspace{-1em}

\begin{align*}
\neg \text{True} &= \text{False} & \neg \neg P &= P \\
P \land Q &= Q \land P & P \lor Q &= Q \lor P \\
(P \land Q) \land R &= P \land (Q \land R) & (P \lor Q) \lor R &= P \lor (Q \lor R) \\
P \land P &= P & P \lor P &= P \\
P \land \neg P &= \text{False} & P \lor \neg P &= \text{True} \\
P \land \text{True} &= P & P \lor \text{False} &= P \\
P \land \text{False} &= \text{False} & P \lor \text{True} &= \text{True}
\end{align*}

\begin{center}
\textbf{Distributive Laws}
\end{center}

\vspace{-2em}

\begin{align*}
P \land (Q \lor R) &= (P \land Q) \lor (P \land R) \\
P \lor (Q \land R) &= (P \lor Q) \land (P \lor R)
\end{align*}

\begin{center}
\textbf{De Morgan's Laws}
\end{center}

\vspace{-2em}

\begin{align*}
\neg (P \lor Q) &= \neg P \land \neg Q \\
\neg (P \land Q) &= \neg P \lor \neg Q
\end{align*}

% Disjunctive Normal Form
\subsection{Disjunctive Normal Form}

\begin{definition}
A boolean expression is in \textbf{disjunctive normal form} if it is a disjunction of one or more conjunctions of literals. This is like the 'expanded' form of a boolean expression.
\end{definition}

\begin{remark}
A \textbf{literal} is a variable or its negation.
\end{remark}

Problem with DNF is that, in real life, logical rules are not usually specified
in a form that is amenable to DNF. They are typically described by listing conditions that
must be satisfied together. 

% Conjunctive Normal Form
\subsection{Conjunctive Normal Form}

When DNF is a disjunction of literals, CNF is simply a conjunction of literals.

\begin{definition}
A boolean expression is in \textbf{conjunctive normal form} if it is a conjunction of one or more disjunctions of literals.
\end{definition}

Each disjunction of literals is called a \textbf{clause}.

% Representing Logical Statements
\subsection{Representing Logical Statements}




\end{document}
